\documentclass{article}
	
\usepackage[margin=1in]{geometry}		% For setting margins
\usepackage{amsmath}				% For Math
\usepackage[]{amssymb}
\usepackage{amsmath}
\usepackage{gensymb}
\usepackage{fancyhdr}				% For fancy header/footer
\usepackage{graphicx}				% For including figure/image
\usepackage{cancel}					% To use the slash to cancel out stuff in work
\usepackage{wasysym}                % For cent symbol
\usepackage{pgfplots}
\usepackage{gnuplottex}
\usepackage{mathtools}

\pgfplotsset{width=7cm,compat=newest}
\usepgfplotslibrary{fillbetween}

% We will externalize the figures
% \usepgfplotslibrary{external}
% \tikzexternalize

%%%%%%%%%%%%%%%%%%%%%%
% Set up fancy header/footer
\pagestyle{fancy}
\fancyhead[RO,R]{{\textbf{Andry Paez}}}
\fancyhead[LO,L]{\textbf{Chapter 14}}
\fancyhead[CO,C]{\textbf{Problem Set 2}}
% \fancyhead[RO,R]{\today}
\fancyfoot[LO,L]{}
\fancyfoot[CO,C]{\thepage}
\fancyfoot[RO,R]{}
\renewcommand{\headrulewidth}{0.4pt}
\renewcommand{\footrulewidth}{0.4pt}
%%%%%%%%%%%%%%%%%%%%%%

\newcommand{\hmwkTitle}{Chapter 14 - Problem Set 2}
% \newcommand{\hmwkDueDate}{February 12, 2014}
\newcommand{\hmwkClass}{Calculus 3}
% \newcommand{\hmwkClassTime}{}
% \newcommand{\hmwkClassInstructor}{Professor Isaac Newton}
\newcommand{\hmwkAuthorName}{\textbf{Andry Paez}}

% math commands
\newcommand{\ihat}{\;\hat{\textbf{\i}}}
\newcommand{\jhat}{\;\hat{\textbf{\j}}}
\newcommand{\khat}{\;\hat{\textbf{k}}}
\newcommand{\rvec}{\vec{r}(t)}
\newcommand{\drvec}{\vec{r}\;'(t)}
\newcommand\vv[1]{\langle #1 \rangle}
\newcommand\vc[2]{\vec{#1}(#2)}
\newcommand\vcd[2]{\vec{#1}\;'(#2)}
\newcommand\vcdd[2]{\vec{#1}\;''(#2)}
\newcommand\vcddd[2]{\vec{#1}\;'''(#2)}
\newcommand\mgv[1]{\|#1\|}
\newcommand\mgvv[2]{\sqrt{\left(#1\right)^2 + \left(#2\right)^2}}
\newcommand\mgvvv[3]{\sqrt{\left(#1\right)^2 + \left(#2\right)^2 + \left(#3\right)^2}}
\newcommand\rr{\quad\Rightarrow\quad}
\newcommand{\limit}[4]{\lim_{(#1, #2) \to (#3, #4)}}
\newcommand{\limi}[2]{\lim_{#1 \to #2}}
\newcommand{\such}{\; | \;}
\newcommand{\lh}{\overset{L'H}{=}}
\newcommand{\solution}{\centerline{\textit{Solution}}}
\newcommand{\pp}[2]{\displaystyle\frac{\partial #1}{\partial #2}}
\newcommand{\dd}[2]{\displaystyle\frac{d #1}{d #2}}
\newcommand{\spc}{\vspace{1em}\hrule\vspace{1em}}
\newcommand{\bp}[1]{\left(#1\right)}
\newcommand{\bb}[1]{\left[#1\right]}
%
% Title Page
%
\title{
    \vspace{3in}
    \textmd{\textbf{\hmwkTitle}}\\
    \vspace{0.5in}
    \textmd{\textbf{\hmwkClass}}\\
    % \normalsize\vspace{0.1in}\small{Due\ on\ \hmwkDueDate\ at 3:10pm}\\
    % \vspace{0.1in}\large{\textit{\hmwkClassInstructor\ \hmwkClassTime}}
    \vspace{4in}
}

\author{\hmwkAuthorName}
\date{}

\begin{document}
\maketitle
\begin{center}
    \section*{\underline{Section 5: The Chain Rule}}
\end{center}
\begin{center}
\subsection*{\textit{3-7 (odd)}}
Use The Chain Rule to find $\dd z t$ or $\dd w t$. 
\end{center}
\subsubsection*{3. $z=xy^3-x^2y,\quad x=t^2+1,\quad y=t^2-1$}
\solution 
\begin{align*}
    \dd z t &= \dd z x \bp{\dd x t} + \dd z y \bp{\dd y t} \\
    \dd z t &= [y^3-2xy](2t) + [3xy^2-x^2](2t) \\
    \Aboxed{\dd z t &= 2t[y^3 - 2xy + 3xy^2 - x^2]}
\end{align*}
\subsubsection*{5. $z=\sin x \cos y,\quad x =\sqrt t, \quad y = 1/t$}
\solution
\begin{align*}
    \dd z t &= \dd z x \bp{\dd x t} + \dd z y \bp{\dd y t} \\
    \dd z t &= [\cos x \cos y]\bp{\frac 1 2 t^{-\frac 1 2}} + [-\sin x \sin y]
    \bp{-t^{-2}} \\
    \Aboxed{\dd z t &= \frac 1 {2\sqrt t} \cos x \cos y + \frac 1 {t^2}\sin x \sin y}
\end{align*}
\subsubsection*{7. $w = xe^{y/z},\quad x=t^2,\quad y=1-t,\quad z=1+2t$}
\solution
\begin{align*}
    \dd w t &= \dd w x \bp{\dd x t} + \dd w y \bp{\dd y t} + \dd w z \bp{\dd z t} \\
    \dd w t &= \bb{e^{y/z}}(2t) + \bb{\frac x z e^{y/z}}(-1) +
    \bb{-\frac {xy}{z^2}e^{y/z}}(2) \\
    \Aboxed{\dd w t &= e^{y/z}\bb{2t - \displaystyle\frac{x}{z} -
    \displaystyle\frac{2xy}{z^2}}}
\end{align*}
\newpage
\begin{center}
\subsection*{\textit{11-15 (odd)}}
Use the Chain Rule to find $\pp z s$ and $\pp z t$
\end{center}
\subsubsection*{11. $z=(x-y)^5,\quad x=s^2t,\quad y=st^2$}
\solution 
\begin{align*}
    \pp z s &= \pp z x \bp{\pp x s} + \pp z y\bp{\pp y s} \\
    \pp z s &= \bb{5(x-y)^4}(2st)+\bb{-5(x-y)^4}(t^2) \\
    \Aboxed{\pp z s &= 5(x-y)^4\bb{2st-t^2}}
\end{align*}
\begin{align*}
    \pp z t &= \pp z x \bp{\pp x t} + \pp z y \bp{\pp y t} \\
    \pp z t &= \bb{5(x-y)^4}(s^2)+\bb{-5(x-y)^4}(2st) \\
    \Aboxed{\pp z t &= 5(x-y)^4\bb{s^2-2st}}
\end{align*}
\subsubsection*{13. $z=\ln (3x+2y),\quad x=s\sin t,\quad y=t\cos s$}
\solution 
\begin{align*}
    \pp z s &= \pp z x \bp{\pp x s} + \pp z y\bp{\pp y s} \\
    \pp z s &= \bb{\displaystyle\frac{3}{3x+2y}}(\sin t) +
    \bb{\displaystyle\frac{2}{3x+2y}}(-t\sin s) \\
    \Aboxed{\pp z s &= \displaystyle\frac{3\sin t - 2t\sin s}{3x+2y}}
\end{align*}
\begin{align*}
    \pp z t &= \pp z x \bp{\pp x t} + \pp z y \bp{\pp y t} \\
    \pp z t &= \bb{\displaystyle\frac{3}{3x+2y}}(s\cos t) +
    \bb{\displaystyle\frac{2}{3x+2y}}(\cos s) \\
    \Aboxed{\pp z t &= \displaystyle\frac{3s\cos t + 2\cos s}{3x+2y}}
\end{align*}
\newpage
\subsubsection*{15. $z=(\sin\theta)/r,\quad r=st,\quad\theta = s^2 + t^2$}
\solution
\begin{align*}
    \pp z s &= \pp z r \bp{\pp r s} + \pp z \theta\bp{\pp \theta s} \\
    \pp z s &= \bb{-\displaystyle\frac{\sin\theta}{r^2}}(t) +
    \bb{\displaystyle\frac{\cos\theta}{r}}(2s) \\
    \Aboxed{\pp z s &= -\displaystyle\frac{t\sin\theta}{r^2} +
    \displaystyle\frac{2s\cos\theta}{r}}
\end{align*}
\begin{align*}
    \pp z t &= \pp z r \bp{\pp r t} + \pp z \theta \bp{\pp \theta t} \\
    \pp z t &=
    \bb{-\displaystyle\frac{\sin\theta}{r^2}}(s)+\bb{\displaystyle\frac{\cos\theta}{r}}(2t)
    \\
    \Aboxed{\pp z t &= -\displaystyle\frac{s\sin\theta}{r^2}+
    \displaystyle\frac{2t\cos\theta}{r}}
\end{align*}
\spc
\begin{center}
\subsection*{\textit{25-29 (odd)}}
Use the Chain Rule to find the indicated partial derivatives.
\end{center}
\subsubsection*{
    25.
    $z = x^4 + x^2y,\quad x = s+2t-u,\quad y =stu^2$;\\\\
    $\pp z s, \pp z t, \pp z u\quad \text{when}\;s=4,t=2,u=1$
}
\solution
\begin{align*}
    \intertext{When $s = 4$, $t=2$, and $u=1$ $\rr$ $x=7$, $y=8$}
    \pp z s &= \bb{4x^3 +2xy}(1) + \bb{x^2}(tu^2) = 4x^3 + 2xy + x^2tu^2 \\
    \pp z s &= 4(7)^3 + 2(7)(8) + (7)^2(2)(1)^2 \\
    \Aboxed{\pp z s &= 1582}
\end{align*}
\begin{align*}
    \pp z t &= \bb{4x^3+2xy}(2)+\bb{x^2}(su^2) = 8x^3+4xy+x^2su^2 \\
    \pp z t &= 8(7)^3+4(7)(8)+(7)^2(4)(1)^2 \\
    \Aboxed{\pp z t &= 3164}
\end{align*}
\begin{align*}
    \pp z u &= \bb{4x^3+2xy}(-1)+\bb{x^2}(2stu) = -4x^3-2xy+2x^2stu \\
    \pp z u &= -4(7)^3-2(7)(8)+2(7)^2(4)(2)(1) \\
    \Aboxed{\pp z u &= -700}
\end{align*}
\subsubsection*{27. 
    $w = xy + yz +zx,\quad x = r\cos\theta,\quad y = r\sin\theta,\quad z =
    r\theta$;\\\\
    $\pp w r, \pp w \theta\quad \text{when}\;r=2,\theta = \pi/2$
}
\solution
\begin{align*}
    \intertext{When $r=2$ and $\theta=\pi/2$ $\rr$ $x=0$, $y=2$, $z=\pi$}
    \pp w r &= [y+z](\cos\theta)+[x+z](\sin\theta)+[x+y](\theta) \\
    \pp w r &= [2+\pi](0)+[0+\pi](1)+[0+2](\pi/2) = 0+\pi+\pi \\
    \Aboxed{\pp w r &= 2\pi}
\end{align*}
\begin{align*}
    \pp w \theta &= [y+z](-r\sin\theta)+[x+z](r\cos\theta)+[x+y](r) \\
    \pp w \theta &= [2+\pi](-2\cdot 1)+[0+\pi](0)+[0+2](2) = -4 - 2\pi + 0 + 4
    \\
    \Aboxed{\pp w \theta &= -2\pi}
\end{align*}
\newpage
\subsubsection*{29. 
    $N = \displaystyle\frac{p+q}{p+r},\quad p=u+vw,\quad q=v+uw,\quad
    r=w+uv$\;\\\\
    $\pp N u, \pp N v, \pp N w\quad \text{when}\; u=2,v=3,w=4$
}
\solution
\begin{align*}
    \intertext{When $u=2$, $v=3$, and $w=4$ $\rr$ $p=14$, $q=11$, $r=10$}
    \pp N u &=
    \bb{\displaystyle\frac{(p+r)-(p+q)}{(p+r)^2}}(1)+\bb{\displaystyle\frac{1}{p+r}}(w)+\bb{-\displaystyle\frac{p+q}{(p+r)^2}}(v)
    \\
    \pp N u &=
    \displaystyle\frac{r-q}{(p+r)^2}+\displaystyle\frac{w}{p+r}-\displaystyle\frac{v(p+q)}{(p+r)^2} \\
    \pp N u &=
    \displaystyle\frac{10-11}{(14+10)^2}+\displaystyle\frac{4}{14+10}-\displaystyle\frac{3(14+11)}{(14+10)^2} \\
    \pp N u &= \displaystyle\frac{-1}{576}+\displaystyle\frac{4}{24}-\displaystyle\frac{75}{576} = \displaystyle\frac{-1+96-75}{576} = \displaystyle\frac{20}{576} \\
    \Aboxed{\pp N u &= \displaystyle\frac{5}{144}}
\end{align*}
\begin{align*}
    \pp N v &=
    \bb{\displaystyle\frac{r-q}{(p+r)^2}}(w)+\bb{\displaystyle\frac{1}{p+r}}(1)+\bb{-\displaystyle\frac{p+q}{(p+r)^2}}(u)
    \\
    \pp N v &=
    \displaystyle\frac{w(r-q)}{(p+r)^2}+\displaystyle\frac{1}{p+r}-\displaystyle\frac{u(p+q)}{(p+r)^2} \\
    \pp N v &=
    \displaystyle\frac{4(10-11)}{(14+10)^2}+\displaystyle\frac{1}{14+10}-\displaystyle\frac{2(14+11)}{(14+10)^2} \\
    \pp N v &= \displaystyle\frac{-4}{576}+\displaystyle\frac{1}{24}-\displaystyle\frac{50}{576}
            = \displaystyle\frac{-4+24-50}{576}=\displaystyle\frac{-30}{576} \\
    \Aboxed{\pp N v &= -\displaystyle\frac{5}{96}}
\end{align*}
\begin{align*}
    \pp N w &=
    \bb{\displaystyle\frac{r-q}{(p+r)^2}}(v)+\bb{\displaystyle\frac{1}{p+r}}(u)+\bb{-\displaystyle\frac{p+q}{(p+r)^2}}(1) \\
    \pp N w &=
    \displaystyle\frac{v(r-q)}{(p+r)^2}+\displaystyle\frac{u}{p+r}-\displaystyle\frac{p+q}{(p+r)^2} \\
    \pp N w &=
    \displaystyle\frac{3(10-11)}{(14+10)^2}+\displaystyle\frac{2}{(14+10)}-\displaystyle\frac{14+11}{(14+10)^2}
    \\
    \pp N w &=
    \displaystyle\frac{-3}{576}+\displaystyle\frac{2}{24}-\displaystyle\frac{25}{576} = \displaystyle\frac{-3+48-25}{576}=\displaystyle\frac{20}{576} \\
    \Aboxed{\pp N w &= \displaystyle\frac{5}{144}}
\end{align*}
\newpage
\begin{center}
    Use Equation 5 to find $\dd y x$ 
\end{center}
\subsubsection*{31. $y\cos x = x^2 + y^2$}
\solution 
\begin{align*}
    \dd y x &= -\displaystyle\frac{\pp F x}{\pp F y} = -\displaystyle\frac{F_x}{F_y}\\
    \intertext{Let $F &= y\cos x -x^2 - y^2$}
    F_x &= -y\sin x -2x \\
    F_y &= \cos x -2y \\
    \Aboxed{\dd y x = -\displaystyle\frac{F_x}{F_y} &=
    -\displaystyle\frac{(-y\sin x -2x)}{\cos x -2y} = \displaystyle\frac{2x+y\sin
x}{\cos x -2y}}
\end{align*}
\spc
\begin{center}
    Use Equations 6 to find $\pp z x$ and $\pp z y$ 
\end{center}
\subsubsection*{35. $x^2+2y^2+3z^2=1$}
\solution
\begin{align*}
    \pp z x = -\displaystyle\frac{F_x}{F_z} &\quad\quad \pp z y = -\displaystyle\frac{F_y}{F_z} \\
    \intertext{Let $F = x^2+&2y^2+3z^2-1$}
    F_x = 2x \quad\quad &F_y = 4y \quad\quad F_z = 6z \\
    \Aboxed{\pp z x = -\displaystyle\frac{F_x}{F_z} &=
    -\displaystyle\frac{2x}{6z} = -\displaystyle\frac{x}{3z}} \\
    \Aboxed{\pp z y = -\displaystyle\frac{F_y}{F_z} &=
    -\displaystyle\frac{4y}{6z} = -\displaystyle\frac{2y}{3z}} \\
\end{align*}
\newpage
\begin{center}
    \section*{\underline{Section 6: Directional Derivatives and the Gradient
    Vector}}
\end{center}
\begin{center}
    \subsection*{\textit{5, 7}}
    Find the directional derivative of $f$ at the given point in the direction
    indicated by the angle $\theta$.
\end{center}
\subsubsection*{5. $f(x,y)=y\cos(xy),\quad(0,1),\quad \theta=\pi / 4$}
\solution 
\begin{align*}
    D_u f(x,y) &= f_x(x,y) \cos\theta + f_y(x,y) \sin\theta \\
    D_u f(x,y) &= \bb{-y^2\sin(xy)}\cos\frac \pi 4 + \bb{\cos(xy) -
    xy\sin(xy)}\sin\frac \pi 4 \\
    D_u f(x,y) &= \displaystyle\frac{\sqrt 2}{2}\bb{-y^2\sin(xy) +
    \cos(xy)-xy\sin(xy)} \\
    D_u f(0,1) &= \displaystyle\frac{\sqrt 2 }{2} \bb{0 + 1 - 0} \\
    \Aboxed{D_u f(0,1) &= \displaystyle\frac{\sqrt 2}{2}}
\end{align*}
\subsubsection*{7. $f(x,y)=arctan(xy),\quad(2,-3),\quad\theta = 3\pi /4$}
\solution
\begin{align*}
    D_u f(x,y) &= f_x(x,y) \cos\theta + f_y(x,y) \sin\theta \\
    D_u f(x,y) &= \bb{\displaystyle\frac{y}{1+(xy)^2}}\cos\frac {3\pi} 4 +
    \bb{\displaystyle\frac{x}{1+(xy)^2}}\sin\frac {3\pi} 4 \\
    D_u f(x,y) &= \frac {\sqrt 2} 2 \bb{\displaystyle\frac{x}{1+(xy)^2} -
    \displaystyle\frac{y}{1+(xy)^2}} \\
    D_u f(2,-3) &= \frac {\sqrt 2} 2 \bb{\displaystyle\frac{2}{1+(2(-3))^2} - \displaystyle\frac{-3}{1+(2(-3))^2}} \\
    D_u f(2,-3) &= \frac {\sqrt 2} 2 \bb{\displaystyle\frac{5}{37}} \\
    \Aboxed{D_u f(2,-3) &= \displaystyle\frac{5\sqrt 2}{74}}
\end{align*}
\newpage
\begin{center}
    \subsection*{\textit{9}} 
\end{center}
\begin{align*}
    &f(x,y)=x/y,\quad P(2,1),\quad \vec u = \frac{3}{5}\ihat + \frac 4 5 \jhat \\
    &\text{(a) Find the gradient of $f$} \\ 
    &\text{(b) Evaluate the gradient at the point $P$} \\
    &\text{(c) Find the rate of change of $f$ at $P$ in the direction of the vector
    $\vec u$}
\end{align*}
\solution 
\begin{align*}
    \intertext{(a)} 
    \nabla f(x,y) &= \vv{f_x(x,y), f_y(x,y)}\quad\text{&or}\quad \pp f x \ihat +
    \pp f y \jhat  \\
    \Aboxed{\nabla f(x,y) &= \vv{\frac 1 y, -\frac x {y^2}}} \\
    \intertext{(b)} 
    \Aboxed{\nabla f(2,1) &= \vv{\frac 1 1, -\frac 2 {1^2}} = \vv{1, -2}}
    \intertext{(c)}
    D_u f(x,y) &= \nabla f(x,y) \cdot \vec u \\
    D_u f(2,1) &= \nabla f(2,1) \cdot \vec u \\ 
    D_u f(2,1) &= \vv{1, -2} \cdot \vv{\frac 3 5, \frac 4 5} \\
    D_u f(2,1) &= \frac 3 5 - \frac 8 5 \\
    \Aboxed{D_u f(2,1) &= -1}
\end{align*}
\newpage 
\begin{center}
    \subsection*{\textit{13, 15}} 
    Find the directional derivative of the function at the given point in the
    direction of the vector $\vec v$.
\end{center}
\subsubsection*{13. $f(x,y)=e^x\sin y,\quad (0,\pi /3),\quad \vec v = \vv{-6, 8}$}
\solution 
\begin{align*}
    \vec u &= \displaystyle\frac{\vec v}{\mgv{\vec v}} = \displaystyle\frac{-6\ihat+8\jhat}{\mgvv{6}{8}} \\
    \vec u &= -\frac {3}{5}\ihat + \frac 4 5\jhat \\
    D_u f(x,y) &= \nabla f(x,y) \cdot \vec u \\
    D_u f(x,y) &= \bp{e^x\sin y\ihat + e^x\cos y\jhat} \cdot \bp{-\frac 3 5\ihat
    + \frac 4 5 \jhat}\\
    D_u f(x,y) &= -\displaystyle\frac{3e^x\sin y}{5} +
    \displaystyle\frac{4e^x\cos y}{5}\\
    D_u f(0,\frac \pi 3) &= -\displaystyle\frac{3(\frac {\sqrt 3} 2)}{5} +
    \displaystyle\frac{4(\frac 1 2)}{5}\\
    \Aboxed{D_u f(0,\frac \pi 3) &= \displaystyle\frac{4-3\sqrt 3}{10}}\\
\end{align*}
\subsubsection*{15. $g(s,t)= s\sqrt t,\quad(2,4),\quad \vec v = 2\ihat - \jhat$}
\solution 
\begin{align*}
    \vec u &= \frac{\vec v}{\mgv{\vec v}} = \frac{2\ihat-\jhat}{\mgvv{2}{-1}} \\ 
    \vec u &= \frac 2 {\sqrt 5}\ihat - \frac 1 {\sqrt 5}\jhat \\
    D_u g(s,t) &= \nabla g(s,t) \cdot \vec u \\
    D_u g(s,t) &= \bp{\sqrt t\ihat + \frac{s}{2\sqrt t}}\cdot\bp{\frac 2 {\sqrt 5}\ihat - \frac 1 {\sqrt 5}\jhat} \\
    D_u g(s,t) &= \displaystyle\frac{2\sqrt t}{\sqrt 5} -
    \displaystyle\frac{s}{2\sqrt t \sqrt 5} \\
    D_u g(s,t) &= \frac 1 {\sqrt 5} \bp{2\sqrt t - \frac s {2\sqrt t}} \\
    D_u g(2,4) &= \frac1 {\sqrt 5}\bp{2\sqrt 4 - \frac 2 {2\sqrt 4}} = \frac 1
    {\sqrt 5}\bp{4 - \frac 1 2} \\
    \Aboxed{D_u g(2,4) &= \frac 1 {\sqrt 5} \bp{\frac {8-1}{2}} = \frac 7 {2\sqrt 5}}
\end{align*}
\newpage
\begin{center}
    \subsection*{\textit{21, 23}} 
    Find the directional derivative of the function at the point $P$ in the
    direction of the point $Q$.
\end{center}
\subsubsection*{21. $f(x,y) = x^2y^2-y^3,\quad P(1,2), \quad Q(-3,5)$}
\solution
\begin{align*}
    \vec v &= \vec {PQ} = \vv{-3-1, 5-2} = \vv{-4, 3} \\
    \vec u &= \frac {\vec v}{\mgv{v}} = \displaystyle\frac{\vv{-4,
    3}}{\sqrt{16+9}} \\
    \vec u &= \vv{-\frac 4 5, \frac 3 5} \\
    D_u f(x,y) &= \nabla f(x,y) \cdot \vec u \\
    D_u f(x,y) &= \vv{2xy^2, 2x^2y-3y^2}\;\cdot\;\vv{-\frac 4 5, \frac 3 5} \\
    D_u f(x,y) &= -\displaystyle\frac{8xy^2}{5} +
    \displaystyle\frac{3(2x^2y-3y^2)}{5} = \frac 1 5(-8xy^2+6x^2y-9y^2) \\
    D_u f(1,2) &= \frac 1 5[-8(1)(4) + 6(1)(2)-9(4)] = \frac 1 5[-32 + 12 - 36] \\
    \Aboxed{D_u f(1,2) &= -\frac{56}{5}}
\end{align*}
\subsubsection*{23. $f(x,y) = \sqrt{xy},\quad P(2,8),\quad Q(5,4)$}
\solution
\begin{align*}
    \vec v &= \vec {PQ} = (5-2)\ihat + (4-8)\jhat = 3\ihat - 4\jhat \\
    \vec u &= \frac {\vec v}{\mgv{v}} = \displaystyle\frac{3\ihat -
    4\jhat}{\sqrt{9 + 16}} \\
    \vec u &= \frac 3 5\ihat - \frac 4 5\jhat \\
    D_u f(x,y) &= \nabla f(x,y) \cdot \vec u \\
    D_u f(x,y) &= 
    \bp{\displaystyle\frac{y}{2\sqrt{xy}}\ihat+\displaystyle\frac{x}{2\sqrt{xy}}\jhat}
    \cdot \bp{\frac 3 5\ihat - \frac 4 5\jhat} \\
    D_u f(x,y) &= \displaystyle\frac{3y}{10\sqrt{xy}} -
    \displaystyle\frac{4x}{10\sqrt{xy}} = \displaystyle\frac{3y-4x}{10\sqrt{xy}}
    \\
    D_u f(2,8) &= \frac{3(8)-4(2)}{10\sqrt{2(8)}} = \frac{24-8}{10\sqrt{16}} =
    \frac {16}{40} \\
    \Aboxed{D_u f(x,y) &= \frac 2 5}
\end{align*}
\newpage 
\begin{center}
    \subsection*{\textit{27, 29}} 
    Find the maximum rate of change of $f$ at the given point and the direction
    in which it occurs.
\end{center}
\subsubsection*{27. $f(x,y) = 5xy^2,\quad (3,-2)$}
\solution
\begin{align*}
    \intertext{The maximum rate of change of f and its direction is given by the
    gradient vector}
    \nabla f(x,y) &= \vv{f_x, f_y} = \vv{5y^2, 10xy} \\
    \nabla f(3,-2) &= \vv{5(-2)^2, 10(3)(-2)} \\
    \intertext{Direction of the maximum rate of change,}
    \Aboxed{\nabla f(3,-2) &= \vv{20, -60} = 20\vv{1,-3}} \\
    \intertext{Maximum rate of change,}
    \Aboxed{\mgv{\nabla f(3,-2)} &= 20\sqrt{1 + 9} = 20\sqrt{10}}
\end{align*}
\subsubsection*{29. $f(x,y) = \sin (xy),\quad (1,0)$}
\solution 
\begin{align*}
    \intertext{The maximum rate of change of f and its direction is given by the
    gradient vector}
    \nabla f(x,y) &= \pp f x\ihat + \pp f y\jhat = y\cos(xy)\ihat +
    x\cos(xy)\jhat \\
    \nabla f(1,0) &= 0 + \jhat \\
    \intertext{Direction of the maximum rate of change,}
    \Aboxed{\nabla f(1,0) &= \jhat} \\
    \intertext{Maximum rate of change,}
    \Aboxed{\mgv{\nabla f(1,0)} &= \sqrt 1 = 1}
\end{align*}
\newpage 
\begin{center}
    \subsection*{\textit{37}} 
\end{center}
The temperature $T$ in a metal ball is inversely proportional to the
distance from the center of the ball, which we take to be the origin. The
temperature at the point $(1,2,2)$ is $120\degree$
\vspace{1em} \\
\vspace{1em}
(a) Find the rate of change of $T$ at $(1,2,2)$ in the direction
toward the point $(2,1,3)$. \\
(b) Show that at any point in the ball the direction of greatest
increase in temperature is given by a vector that points toward the origin

\solution 
\begin{align*}
    \intertext{Since $T \propto \frac 1 d$, $T(x,y,z) =
    \displaystyle\frac{c}{\sqrt{x^2+y^2+z^2}}$, for some constant c, and Euclidean distance to describe the distance from the origin.}
    \intertext{If $120 = T(1,2,2) = \displaystyle\frac c {\sqrt{1^2+2^2+2^2}} = \frac c 3$, then $c = 360$}
    \intertext{$\therefore$ Our function for temperature is,}
    T(x,y,z) &= \displaystyle\frac{360}{\sqrt{x^2+y^2+z^2}} \\
    \intertext{(a)}
    \vec v &= \vv{2-1, 1-2, 3-2} = \vv{1,-1,1} \\
    \vec u &= \frac {\vec v}{\mgv{\vec v}} = \frac {\vv{1,-1,1}}{\sqrt{1+1+1}}
    \\
    \vec u &= \vv{\frac 1 {\sqrt 3}, -\frac 1 {\sqrt 3}, \frac 1 {\sqrt 3}} \\
    \intertext{Getting our gradient vector,}
    \nabla T(x,y,z) &= \vv{T_x, T_y, T_z} \\
    \nabla T(x,y,z) = \vv{-180\cdot
    2x(x^2+y^2+z^2)^{-3/2}&, -180\cdot 2y(x^2+y^2+z^2)^{-3/2}, -180\cdot 2z(x^2+y^2+z^2)^{-3/2}} \\
    \nabla T(x,y,z) &= \displaystyle\frac{360}{(x^2+y^2+z^2)^{3/2}}\vv{-x,-y,-z}
    \intertext{so, our rate of change is}
    D_u T(1,2,2) &= \nabla T(1,2,2) \cdot \vec u \\
    D_u T(1,2,2) &= \frac{360}{(1+4+4)^{3/2}}\vv{-1,-2,-2} \cdot \vv{\frac 1
    {\sqrt 3}, -\frac 1 {\sqrt 3}, \frac 1 {\sqrt 3}} \\
    D_u T(1,2,2) &= \frac {360}{27}\bp{-\frac 1 {\sqrt 3} + \frac 2 {\sqrt
    3} - \frac 2 {\sqrt 3}} = \frac {360}{27}\bp{-\frac 1 {\sqrt 3}} \\
    \Aboxed{D_u T(1,2,2) &= \frac {40} 3 \bp{-\frac 1 {\sqrt 3}} = -\frac {40}{3\sqrt 3}} \\
    \intertext{(b) The direction of greatest increase at any point is the
    direction of $\nabla T$ at that point. In part (a) we saw that $\nabla
    T(x,y,z)$ and $\vv{-x,-y,-z}$ have the same direction and the vector
    $\vv{-x,-y,-z}$ points towards the origin.}
\end{align*}
\newpage
\begin{center}
    \subsection*{\textit{47-51 (odd)}} 
    Find equations of (a) the tangent plane and (b) the normal line to the given
    surface at the specified point.
\end{center}
\subsubsection*{47. $2(x-2)^2 + (y-1)^2 + (z-3)^2 = 10,\quad (3,3,5)$}
\solution 
\begin{align*}
    \intertext{(a)}
    F(x,y,z) &= 2(x-2)^2 + (y-1)^2 + (z-3)^2 - 10 \\\\
    F_x(3,3,5) &= 4(x-2)=4(1)=4 \\ 
    F_y(3,3,5) &= 2(y-1)=2(2)=4 \\ 
    F_z(3,3,5) &= 2(z-3)=2(2)=4 \\
    \intertext{The equation of the tangent plane is given by,}
    F_x(x_0,y_0,z_0)(x-x_0)&+F_y(x_0,y_0,z_0)(y-y_0)+F_z(x_0,y_0,z_0)(z-z_0)
    \\\\
    4(x-3)&+4(y-3)+4(z-5) = 0 \\
    x-3 &+ y-3 + z-5 = 0 \\
    \Aboxed{x &+ y + z = 11}
    \intertext{(b) The equation of the normal line is given by,}
    \displaystyle\frac{x-x_0}{F_x(x_0,y_0,z_0)} &= \displaystyle\frac{y-y_0}{F_y(x_0,y_0,z_0)} =\displaystyle\frac{z-z_0}{F_z(x_0,y_0,z_0)} \\\\
    \frac {x-3} 4 &= \frac {y-3} 4 = \frac {z-5} 4 \\
    \Aboxed{x-3&=y-3=z-5} \\
\end{align*}
\subsubsection*{49. $xy^2z^3=8,\quad (2,2,1)$}
\solution
\begin{align*}
    \intertext{(a)}
    F(x,y,z) &= xy^2z^3 - 8 \\\\
    F_x(2,2,1) &= y^2z^3 = (4)(1) = 4 \\
    F_y(2,2,1) &= 2xyz^3 = 2(2)(2)(1) = 8 \\
    F_z(2,2,1) &= 3xy^2z^2 = 3(2)(4)(1) = 24 \\
    \intertext{The equation of the tangent plane is given by,}
    F_x(x_0,y_0,z_0)(x-x_0)&+F_y(x_0,y_0,z_0)(y-y_0)+F_z(x_0,y_0,z_0)(z-z_0)
    \\\\
    4(x-2)&+8(y-2) + 24(z-1) = 0 \\
    x-2&+2(y-2) + 6(z-1) = 0 \\
    x-2 &+ 2y-4 + 6z-6 = 0 \\
    \Aboxed{x&+2y+6z=12}
    \intertext{(b) The equation of the normal line is given by,}
    \displaystyle\frac{x-x_0}{F_x(x_0,y_0,z_0)} &= \displaystyle\frac{y-y_0}{F_y(x_0,y_0,z_0)} =\displaystyle\frac{z-z_0}{F_z(x_0,y_0,z_0)} \\\\
    \frac {x-2} 4 &= \frac {y-2} 8 = \frac {z-1}{24} \\
    \Aboxed{x-2&=\frac {y-2} 2 =\frac {z-1} 6}
\end{align*}
\subsubsection*{51. $x+y+z=e^{xyz},\quad (0,0,1)$}
\solution
\begin{align*}
    \intertext{(a)}
    F(x,y,z) &= x + y + z - e^{xyz} \\\\
    F_x(0,0,1) &= 1-yze^{xyz} = 1-0=1 \\
    F_y(0,0,1) &= 1-xze^{xyz} = 1-0=1 \\
    F_z(0,0,1) &= 1-xye^{xyz} = 1-0=1 \\
    \intertext{The equation of the tangent plane is given by,}
    F_x(x_0,y_0,z_0)(x-x_0)&+F_y(x_0,y_0,z_0)(y-y_0)+F_z(x_0,y_0,z_0)(z-z_0)
    \\\\
    1(x-0) &+ 1(y-0)+ 1(z-1) = 0 \\
    \Aboxed{x &+ y + z = 1}
    \intertext{(b) The equation of the normal line is given by,}
    \displaystyle\frac{x-x_0}{F_x(x_0,y_0,z_0)} &= \displaystyle\frac{y-y_0}{F_y(x_0,y_0,z_0)} =\displaystyle\frac{z-z_0}{F_z(x_0,y_0,z_0)} \\\\
    \frac {x-0} 1 &= \frac {y-0} 1 = \frac {z-1} 1 \\
    \Aboxed{x&=y=z-1}
\end{align*}
\newpage
\begin{center}
    \section*{\underline{Section 7: Maximum and Minimum Values}}
\end{center}
\begin{center}
    \subsection*{\textit{5-21 (odd)}} 
\end{center}
Find the local maximum and minimum values and saddle point(s) of the
function. You are encouraged to use a calculator or computer to graph the
function with a domain and viewpoint that reveals all the important aspects
of the function.
\subsubsection*{5. $f(x,y)=x^2 + xy + y^2 + y$}
\solution 
\begin{align*}
    \nabla f(x,y) &= \vv{2x+y,\;x+2y+1}
    \intertext{Finding critical values, (system of two equations)}
    2x+y &=0 \\
    x+2y &=-1 \\\\
    2x+y &=0 \\
    -2x-4y &=2 \\
    -3y &=2 \\
    y &=-\frac 2 3 \rr x = \frac 1 3
\end{align*}
\intertext{Critical point at $\bp{\frac 1 3, -\frac 2 3}$} \\\\
\intertext{Hessian matrix to determine maxima and minima,}
\[
    f_{xx}=2,\quad f_{yy}=2,\quad f_{xy}=1
\]
\begin{align*}
    H_f(x,y) &= det\begin{vmatrix}
        f_{xx} & f_{xy} \\\\
        f_{yx} & f_{yy} \\
    \end{vmatrix} \\
        H_f\bp{\frac 1 3,\frac {-2}{3}} &= det\begin{vmatrix}
        2 & 1 \\\\
        1 & 2 \\
    \end{vmatrix} = 4 - 1 = 3 \\
            H_f\bp{\frac 1 3, -\frac 2 3} > 0\;&\text{and}\;f_{xx} = 2 > 0 \\
\end{align*}
\intertext{\Aboxed{\therefore f\bp{\frac 1 3, -\frac 2 3} \text{is a local minimum}}
\subsubsection*{7. $f(x,y)= 2x^2-8xy+y^4-4y^3$}
\solution 
\begin{align*}
    \nabla f(x,y) &= \vv{4x-8y,\;-8x+4y^3-12y^2} 
    \intertext{Finding critical values,}
    0=&=4x-8y\\
    2x&=8y\\
    x&=2y\\\\
    0&=-8x+4x^3-12y^2\\
    0&=-8(2y)+4y^3-12y^2\\
    0&=4y^3-12y^2-16y\\
    0&=4y(y^2-3y-4)\\
    0&=4y(y-4)(y+1)\\
    y=-1,0,4&\rr x=-2,0,8
\end{align*}
\intertext{Critical points at $\bp{-2,-1},\bp{0,0},\bp{8,4}$}
\[
    f_{xx} = 4,\quad f_{yy} = 12y^2-24y,\quad f_{xy} = -8
\]
\begin{align*}
    H_f(8,4) &= det\begin{vmatrix}
        4 & -8 \\\\ 
        -8 & 96 \\
    \end{vmatrix} =384 - 64 = 320 \\
    H_f(-2,-1) &= det\begin{vmatrix}
        4 & -8 \\\\ 
        -8 & 36 \\
    \end{vmatrix} =144 - 64 = 80 \\
        H_f(0,0) &= det\begin{vmatrix}
        4 & -8 \\\\ 
        -8 & 0 \\
    \end{vmatrix} =0-64=-64 \\
\end{align*}
\intertext{$H_f(8,4) > 0$ and $f_{xx}=4>0$ so f(8,4) is a local minima} \\
\intertext{$H_f(-2,-1) > 0$ and $f_{xx}=4>0$ so f(8,4) is a local minima} \\
\intertext{$H_f(0,0)$ so f(8,4) is a saddle point} \\\\
\intertext{\Aboxed{\text{Minima: }f(8,4=-128),\quad f(-2,-1)=3}} \\
\intertext{\Aboxed{\text{Saddle Point: }f(0,0)}}
\subsubsection*{9. $f(x,y)=(x-y)(1-xy)$}
\solution
\[
    \nabla f(x,y) &= \vv{1-2xy+y^2,\; -x^2-1+2xy}
\]
\intertext{Finding critical values,}
\begin{align*}
    0 &= 1-2xy+y^2 \\
    0 &= 1-y(2x+y) \\
    1 &= y(2x+y) \\
    y &= 1,\quad 2x+y=1 \rr y=1-2x \\\\
    0 &= -x^2-1+2xy \\
    0 &= -x^2-1+2x(1-2x) \\
    0 &= -x^2-1+2x-4x \\
    0 &= -x^2-2x-1 \\
    0 &= x^2+2x+1 \\
    0 &= (x+1)(x+1) \\
    x&= -1 \rr y= 1 - 2(-1) = -1 \rr (-1,-1)\\ 
\end{align*}
\begin{align*}
    \intertext{If $y=1$,}
    0 &= -x^2+2x-1 \\
    0 &= x - 2x + 1 \\
    0 &= (x-1)(x-1) \\
    x &= 1 \rr (1,1)
\end{align*}
\intertext{Critical points at (-1, -1), (1, 1)} \\
\[
    f_{xx} = -2y,\quad f_{yy} = 2x,\quad f_{xy} = -2x + 2y
\]
\begin{align*}
    H_f(-1,-1) &= det\begin{vmatrix}
    2 & 0 \\\\ 
    0 & -2 \\
    \end{vmatrix} = -4 - 0 = -4 \\
    H_f(1,1) &= det\begin{vmatrix}
    -2 & 0 \\\\ 
    0 & 2 \\
    \end{vmatrix} = -4 - 0 = -4 \\
\end{align*}
\intertext{$H_f(-1,-1)$ and $H_f(1,1)$ are both $<$ 0, so they are saddle points.} 
\\\\
\intertext{\Aboxed{\text{Saddle Points: }f(-1, -1), f(1, 1)}}
\subsubsection*{11. $f(x,y)=y\sqrt x -y^2-2x+7y$}
\solution
\subsubsection*{13. $f(x,y)=x^3-3x+3xy^2$}
\subsubsection*{15. $f(x,y)=x^4 - 2x^2+y^3-3y$}
\subsubsection*{17. $f(x,y)=xy-x^2y-xy^2$}
\subsubsection*{19. $f(x,y)=e^x\cos y$}
\subsubsection*{21. $f(x,y)=y^2-2y\cos x,\quad -1 \leq x \leq 7$}
\spc
\begin{center}
    \subsection*{\textit{33-39 (odd)}} 
    Find the absolute maximum and minimum values of $f$ on the set $D$.
\end{center}
\subsubsection*{33. $f(x,y) = x^2+y^2-2x,\quad$}
$D$ is the closed triangular
region with vertices $(2,0)$, $(0,2)$, and $(0,-2)$
\subsubsection*{35. $f(x,y)=x^2+y^2+x^2y+4,$} 
$D = \{(x,y)\such |x| \leq 1,\; |y| \leq 1\}$
\subsubsection*{37. $f(x,y)=x^2+2y^2-2x-4y+1,$}
$D = \{(x,y)\such 0\leq x \leq 2,\;0 \leq y \leq 3\}$
\subsubsection*{39. $f(x,y)= 2x^3+y^4,$}
$D=\{(x,y)\such x^2+y^2 \leq 1\}
\spc 
\begin{center}
    \subsection*{\textit{43}} 
Find the shortest distance from the point $(2,0,-3)$ to the plane $x+y+z=1$. 
\end{center}
\spc
\begin{center}
    \subsection*{\textit{45}} 
Find the points on the cone $z^2 = x^2 + y^2$ that are closest to the point
$(4,2,0)$. 
\end{center}
\spc
\begin{center}
    \subsection*{\textit{47}} 
Find three positive numbers whose sum is 100 and whose product is a maximum.
\end{center}
\spc
\begin{center}
    \subsection*{\textit{55}} 
\end{center}
A cardboard box without a lid is to have a volume of 32,000 $cm^3$. Find the
dimensions that minimize the amount of cardboard used.
\newpage
\begin{center}
    \section*{\underline{Section 8: Lagrange Multipliers}}
\end{center}
\begin{center}
    \subsection*{\textit{3-13 (odd)}} 
\end{center}
Each of these extreme value problems has a solution with both a maximum
value and minimum value. Use Lagrange multipliers to find the extreme values
of the function subject to the given constraint. 
\subsubsection*{3. $f(x,y) = x^2 -y^2,\quad x^2+y^2=1$}
\subsubsection*{5. $f(x,y) = xy,\quad 4x^2+y^2=8$}
\subsubsection*{7. $f(x,y) = 2x^2+6y^2,\quad x^4+3y^4=1$}
\subsubsection*{9. $f(x,y, z) = 2x+2y+z,\quad x^2+y^2+z^2 = 9$}
\subsubsection*{11. $f(x,y, z) = xy^2z,\quad x^2+y^2+z^2=4$}
\subsubsection*{13. $f(x,y, z) = x^2+y^2+z^2,\quad x^4+y^4+z^4=1$}
\spc
\begin{center}
    \subsection*{\textit{23}} 
\end{center}
The method of Lagrange multipliers assumes that the extreme values exist,
but that is not always the case. Show that the problem of finding the
minimum value of $f$ subject to the given constraint can be solved using
Lagrange multipliers, but $f$ does not have a maximum value with that constraint.
\[
f(x,y) = x^2+y^2,\quad xy=1
\]
\spc
\begin{center}
    \subsection*{\textit{25}} 
\end{center}
Use Lagrange multipliers to find the maximum value of $f$ subject to the given
constraint. Then show that $f$ has no minimum value with that constraint.
\[
    f(x,y) = e^{xy},\quad x^3+y^3=16
\]
\spc
\begin{center}
    \subsection*{\textit{27, 29}} 
    Find the extreme values of f on the region described by the inequality.
\end{center}
\subsubsection*{27. $f(x,y) = x^2+y^2+4x-4y,\quad x^2+y^2 \leq 9$}
\subsubsection*{29. $f(x,y) = e^{-xy},\quad x^2+4y^2 \leq 1$}
\spc
\begin{center}
    \subsection*{\textit{31, 33}} 
    Find the extreme values of $f$ subject to both constraints
\end{center}
\subsubsection*{31. $f(x,y, z) = x+y+z;\quad x^2+z^2=2,\quad x+y=1$}
\subsubsection*{33. $f(x,y, z) = yz+xy;\quad xy=1,\quad y^2+z^2=1$}
\end{document}
