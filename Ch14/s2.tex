\documentclass{article}
	
\usepackage[margin=1in]{geometry}		% For setting margins
\usepackage{amsmath}				% For Math
\usepackage[]{amssymb}
\usepackage{amsmath}
\usepackage{gensymb}
\usepackage{fancyhdr}				% For fancy header/footer
\usepackage{graphicx}				% For including figure/image
\usepackage{cancel}					% To use the slash to cancel out stuff in work
\usepackage{wasysym}                % For cent symbol
\usepackage{pgfplots}
\usepackage{gnuplottex}

\pgfplotsset{width=7cm,compat=newest}
\usepgfplotslibrary{fillbetween}

% We will externalize the figures
% \usepgfplotslibrary{external}
% \tikzexternalize

%%%%%%%%%%%%%%%%%%%%%%
% Set up fancy header/footer
\pagestyle{fancy}
\fancyhead[RO,R]{{\textbf{Andry Paez}}}
\fancyhead[LO,L]{\textbf{Chapter 14}}
\fancyhead[CO,C]{\textbf{Problem Set 2}}
% \fancyhead[RO,R]{\today}
\fancyfoot[LO,L]{}
\fancyfoot[CO,C]{\thepage}
\fancyfoot[RO,R]{}
\renewcommand{\headrulewidth}{0.4pt}
\renewcommand{\footrulewidth}{0.4pt}
%%%%%%%%%%%%%%%%%%%%%%

\newcommand{\hmwkTitle}{Chapter 14 - Problem Set 2}
% \newcommand{\hmwkDueDate}{February 12, 2014}
\newcommand{\hmwkClass}{Calculus 3}
% \newcommand{\hmwkClassTime}{}
% \newcommand{\hmwkClassInstructor}{Professor Isaac Newton}
\newcommand{\hmwkAuthorName}{\textbf{Andry Paez}}

% math commands
\newcommand{\ihat}{\;\hat{\textbf{\i}}}
\newcommand{\jhat}{\;\hat{\textbf{\j}}}
\newcommand{\khat}{\;\hat{\textbf{k}}}
\newcommand{\rvec}{\vec{r}(t)}
\newcommand{\drvec}{\vec{r}\;'(t)}
\newcommand\vv[1]{\langle #1 \rangle}
\newcommand\vc[2]{\vec{#1}(#2)}
\newcommand\vcd[2]{\vec{#1}\;'(#2)}
\newcommand\vcdd[2]{\vec{#1}\;''(#2)}
\newcommand\vcddd[2]{\vec{#1}\;'''(#2)}
\newcommand\mgv[1]{\|#1\|}
\newcommand\mgvv[2]{\sqrt{\left(#1\right)^2 + \left(#2\right)^2}}
\newcommand\mgvvv[3]{\sqrt{\left(#1\right)^2 + \left(#2\right)^2 + \left(#3\right)^2}}
\newcommand\rr{\quad\Rightarrow\quad}
\newcommand{\limit}[4]{\lim_{(#1, #2) \to (#3, #4)}}
\newcommand{\limi}[2]{\lim_{#1 \to #2}}
\newcommand{\such}{\; | \;}
\newcommand{\lh}{\overset{L'H}{=}}
\newcommand{\solution}{\centerline{\textbf{Solution}}}
\newcommand{\pp}[2]{\displaystyle\frac{\partial #1}{\partial #2}}
\newcommand{\dd}[2]{\displaystyle\frac{d #1}{d #2}}
\newcommand{\spc}{\vspace{1em}\hrule\vspace{1em}}
%
% Title Page
%
\title{
    \vspace{3in}
    \textmd{\textbf{\hmwkTitle}}\\
    \vspace{0.5in}
    \textmd{\textbf{\hmwkClass}}\\
    % \normalsize\vspace{0.1in}\small{Due\ on\ \hmwkDueDate\ at 3:10pm}\\
    % \vspace{0.1in}\large{\textit{\hmwkClassInstructor\ \hmwkClassTime}}
    \vspace{4in}
}

\author{\hmwkAuthorName}
\date{}

\begin{document}
\maketitle
\begin{center}
    \section*{\underline{Section 5: The Chain Rule}}
\end{center}
\begin{center}
\subsection*{\textit{3-7 (odd)}}
Use The Chain Rule to find $\dd z t$ or $\dd w t$. 
\end{center}
\subsubsection*{3. $z=xy^3-x^2y,\quad x=t^2+1,\quad y=t^2-1$}
\subsubsection*{5. $z=\sin x \cos y,\quad x =\sqrt t, \quad y = 1/t$}
\subsubsection*{7. $w = xe^{y/z},\quad x=t^2,\quad y=1-t,\quad z=1+2t$}
\spc
\begin{center}
\subsection*{\textit{11-15 (odd)}}
Use the Chain Rule to find $\pp z s$ and $\pp z t$
\end{center}
\subsubsection*{11. $z=(x-y)^5,\quad x=s^2t,\quad y=st^2$}
\subsubsection*{13. $z=\ln (3x+2y),\quad x=s\sin t,\quad y=t\cos s$}
\subsubsection*{15. $z=(\sin\theta)/r,\quad r=st,\quad\theta = s^2 + t^2$}
\begin{center}
\spc
\subsection*{\textit{25-29 (odd)}}
Use the Chain Rule to find the indicated partial derivatives.
\end{center}
\subsubsection*{
    25.
    $z = x^4 + x^2y,\quad x = s+2t-u,\quad y =stu^2$;\\\\
    $\pp z s, \pp z t, \pp z u\quad \text{when}\;s=4,t=e,u=1$
}
\subsubsection*{27. 
    $w = xy + yz +zx,\quad x = r\cos\theta,\quad y = r\cos\theta,\quad z =
    r\theta$;\\\\
    $\pp w r, \pp w \theta\quad \text{when}\;r=2,\theta = \pi/2$
}
\subsubsection*{29. 
    $N = \displaystyle\frac{p+q}{p+r},\quad p=u+vw,\quad q=v+uw,\quad
    r=w+uv$\;\\\\
    $\pp N u, \pp N v, \pp N w\quad \text{when}\; u=2,v=3,w=4$
}
\spc
\begin{center}
    Use Equation 5 to find $\dd y x$ 
\end{center}
\subsubsection*{31. $y\cos x = x^2 + y^2$}
\spc
\begin{center}
    Use Equations 6 to find $\pp z x$ and $\pp z y$ 
\end{center}
\subsubsection*{35. $x^2+2y^2+3z^2=1$}
\newpage
\begin{center}
    \section*{\underline{Section 6: Directional Derivatives and the Gradient
    Vector}}
\end{center}
\begin{center}
    \subsection*{\textit{5, 7}}
    Find the directional derivative of $f$ at the given point in the direction
    indicated by the angle $\theta$.
\end{center}
\subsubsection*{5.}
\subsubsection*{7.}
\spc 
\begin{align*}
    &\text{(a) Find the gradient of $f$} \\ 
    &\text{(b) Evaluate the gradient at the point $P$} \\
    &\text{(c) Find the rate of change of $f$ at $P$ in the direction of the vector
    $\vec u$}
\end{align*}
\subsubsection*{9.}
\spc 
\begin{center}
    \subsection*{\textit{13, 15}} 
    Find the directional derivative of the function at the given point in the
    direction of the vector $\vec v$.
\end{center}
\subsubsection*{13.}
\subsubsection*{15.}
\spc
\begin{center}
    \subsection*{\textit{21, 23}} 
    Find the directional derivative of the function at the point $P$ in the
    direction of the point $Q$.
\end{center}
\subsubsection*{21.}
\subsubsection*{23.}
\spc 
\begin{center}
    \subsection*{\textit{27, 29}} 
    Find the maximum rate of change of $f$ at the given point and the direction
    in which it occurs.
\end{center}
\subsubsection*{27.}
\subsubsection*{29.}
\spc 
\begin{align*}
    \intertext{\subsubsection*{37. }The temperature $T$ in a metal ball is inversely proportional to the
    distance from the center of the ball, which we take to be the origin. The
    temperature at the point $(1,2,2)$ is $120\degree$}
    \intertext{(a) Find the rate of change of $T$ at $(1,2,2)$ in the direction
    toward the point $(2,1,3)$.}
    \intertext{(b) Show that at any point in the ball the direction of greatest
    increase in temperature is given by a vector that points toward the origin}
 \end{align*}
 \spc
\begin{center}
    \subsection*{\textit{47-51 (odd)}} 
    Find equations of (a) the tangent plane and (b) the normal line to the given
    surface at the specified point.
\end{center}
\subsubsection*{47.}
\subsubsection*{49.}
\subsubsection*{51.}
\newpage
\begin{center}
    \section*{\underline{Section 7: Maximum and Minimum Values}}
\end{center}
\begin{center}
    \subsection*{\textit{5-21 (odd)}} 
\end{center}
\begin{align*}
    \intertext{Find the local maximum and minumum values and saddle point(s) of the
    function. You are encouraged to use a calculator or computer to graph the
    function with a domain and viewpoint that reveals all the important aspects
    of the function.}
\end{align*}
\subsubsection*{5.}
\subsubsection*{7.}
\subsubsection*{9.}
\subsubsection*{11.}
\subsubsection*{13.}
\subsubsection*{15.}
\subsubsection*{17.}
\subsubsection*{19.}
\subsubsection*{21.}
\spc
\begin{center}
    \subsection*{\textit{33-39 (odd)}} 
    Find the absolute maximum and minimum values of $f$ on the set $D$.
\end{center}
\subsubsection*{33.}
\subsubsection*{35.}
\subsubsection*{37.}
\subsubsection*{39.}
\spc 
\begin{center}
    \subsection*{\textit{43}} 
Find the shortest distance from the point $(2,0,-3)$ to the plan $x+y+z=1$. 
\end{center}
\spc
\begin{center}
    \subsection*{\textit{45}} 
Find the points on the cone $z^2 = x^2 + y^2$ that are closest to the point
$(4,2,0)$. 
\end{center}
\spc
\begin{center}
    \subsection*{\textit{47}} 
Find three positive numbers whose sum is 100 and whose product is a maximum.
\end{center}
\spc
\begin{center}
    \subsection*{\textit{55}} 
    A cardboard box without a lid is to have a volume of 32,000 $cm^3$. Find the
    dimensions that minimize the amount of cardboard used.
\end{center}
\end{document}
