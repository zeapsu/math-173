\documentclass{article}
	
\usepackage[margin=1in]{geometry}		% For setting margins
\usepackage{amsmath}				% For Math
\usepackage[]{amssymb}
\usepackage{amsmath}
\usepackage{gensymb}
\usepackage{fancyhdr}				% For fancy header/footer
\usepackage{graphicx}				% For including figure/image
\usepackage{cancel}					% To use the slash to cancel out stuff in work
\usepackage{wasysym}                % For cent symbol
\usepackage{pgfplots}
\usepackage{gnuplottex}
\usepackage{mathtools}

\pgfplotsset{width=7cm,compat=newest}
\usepgfplotslibrary{fillbetween}

% We will externalize the figures
% \usepgfplotslibrary{external}
% \tikzexternalize

%%%%%%%%%%%%%%%%%%%%%%
% Set up fancy header/footer
\pagestyle{fancy}
\fancyhead[RO,R]{{\textbf{Andry Paez}}}
\fancyhead[LO,L]{\textbf{Chapter 14}}
\fancyhead[CO,C]{\textbf{Problem Set 2}}
% \fancyhead[RO,R]{\today}
\fancyfoot[LO,L]{}
\fancyfoot[CO,C]{\thepage}
\fancyfoot[RO,R]{}
\renewcommand{\headrulewidth}{0.4pt}
\renewcommand{\footrulewidth}{0.4pt}
%%%%%%%%%%%%%%%%%%%%%%

\newcommand{\hmwkTitle}{Chapter 14 - Problem Set 2}
% \newcommand{\hmwkDueDate}{February 12, 2014}
\newcommand{\hmwkClass}{Calculus 3}
% \newcommand{\hmwkClassTime}{}
% \newcommand{\hmwkClassInstructor}{Professor Isaac Newton}
\newcommand{\hmwkAuthorName}{\textbf{Andry Paez}}

% math commands
\newcommand{\ihat}{\;\hat{\textbf{\i}}}
\newcommand{\jhat}{\;\hat{\textbf{\j}}}
\newcommand{\khat}{\;\hat{\textbf{k}}}
\newcommand{\rvec}{\vec{r}(t)}
\newcommand{\drvec}{\vec{r}\;'(t)}
\newcommand\vv[1]{\langle #1 \rangle}
\newcommand\vc[2]{\vec{#1}(#2)}
\newcommand\vcd[2]{\vec{#1}\;'(#2)}
\newcommand\vcdd[2]{\vec{#1}\;''(#2)}
\newcommand\vcddd[2]{\vec{#1}\;'''(#2)}
\newcommand\mgv[1]{\|#1\|}
\newcommand\mgvv[2]{\sqrt{\left(#1\right)^2 + \left(#2\right)^2}}
\newcommand\mgvvv[3]{\sqrt{\left(#1\right)^2 + \left(#2\right)^2 + \left(#3\right)^2}}
\newcommand\rr{\quad\Rightarrow\quad}
\newcommand{\limit}[4]{\lim_{(#1, #2) \to (#3, #4)}}
\newcommand{\limi}[2]{\lim_{#1 \to #2}}
\newcommand{\such}{\; | \;}
\newcommand{\lh}{\overset{L'H}{=}}
\newcommand{\solution}{\centerline{\textit{Solution}}}
\newcommand{\pp}[2]{\displaystyle\frac{\partial #1}{\partial #2}}
\newcommand{\dd}[2]{\displaystyle\frac{d #1}{d #2}}
\newcommand{\spc}{\vspace{1em}\hrule\vspace{1em}}
\newcommand{\bp}[1]{\left(#1\right)}
\newcommand{\bb}[1]{\left[#1\right]}
%
% Title Page
%
\title{
    \vspace{3in}
    \textmd{\textbf{\hmwkTitle}}\\
    \vspace{0.5in}
    \textmd{\textbf{\hmwkClass}}\\
    % \normalsize\vspace{0.1in}\small{Due\ on\ \hmwkDueDate\ at 3:10pm}\\
    % \vspace{0.1in}\large{\textit{\hmwkClassInstructor\ \hmwkClassTime}}
    \vspace{4in}
}

\author{\hmwkAuthorName}
\date{}

\begin{document}
\maketitle
\begin{center}
    \section*{\underline{Section 5: The Chain Rule}}
\end{center}
\begin{center}
\subsection*{\textit{3-7 (odd)}}
Use The Chain Rule to find $\dd z t$ or $\dd w t$. 
\end{center}
\subsubsection*{3. $z=xy^3-x^2y,\quad x=t^2+1,\quad y=t^2-1$}
\solution 
\begin{align*}
    \dd z t &= \dd z x \bp{\dd x t} + \dd z y \bp{\dd y t} \\
    \dd z t &= [y^3-2xy](2t) + [3xy^2-x^2](2t) \\
    \Aboxed{\dd z t &= 2t[y^3 - 2xy + 3xy^2 - x^2]}
\end{align*}
\subsubsection*{5. $z=\sin x \cos y,\quad x =\sqrt t, \quad y = 1/t$}
\solution
\begin{align*}
    \dd z t &= \dd z x \bp{\dd x t} + \dd z y \bp{\dd y t} \\
    \dd z t &= [\cos x \cos y]\bp{\frac 1 2 t^{-\frac 1 2}} + [-\sin x \sin y]
    \bp{-t^{-2}} \\
    \Aboxed{\dd z t &= \frac 1 {2\sqrt t} \cos x \cos y + \frac 1 {t^2}\sin x \sin y}
\end{align*}
\subsubsection*{7. $w = xe^{y/z},\quad x=t^2,\quad y=1-t,\quad z=1+2t$}
\solution
\begin{align*}
    \dd w t &= \dd w x \bp{\dd x t} + \dd w y \bp{\dd y t} + \dd w z \bp{\dd z t} \\
    \dd w t &= \bb{e^{y/z}}(2t) + \bb{\frac x z e^{y/z}}(-1) +
    \bb{-\frac {xy}{z^2}e^{y/z}}(2) \\
    \Aboxed{\dd w t &= e^{y/z}\bb{2t - \displaystyle\frac{x}{z} -
    \displaystyle\frac{2xy}{z^2}}}
\end{align*}
\newpage
\begin{center}
\subsection*{\textit{11-15 (odd)}}
Use the Chain Rule to find $\pp z s$ and $\pp z t$
\end{center}
\subsubsection*{11. $z=(x-y)^5,\quad x=s^2t,\quad y=st^2$}
\solution 
\begin{align*}
    \pp z s &= \pp z x \bp{\pp x s} + \pp z y\bp{\pp y s} \\
    \pp z s &= \bb{5(x-y)^4}(2st)+\bb{-5(x-y)^4}(t^2) \\
    \Aboxed{\pp z s &= 5(x-y)^4\bb{2st-t^2}}
\end{align*}
\begin{align*}
    \pp z t &= \pp z x \bp{\pp x t} + \pp z y \bp{\pp y t} \\
    \pp z t &= \bb{5(x-y)^4}(s^2)+\bb{-5(x-y)^4}(2st) \\
    \Aboxed{\pp z t &= 5(x-y)^4\bb{s^2-2st}}
\end{align*}
\subsubsection*{13. $z=\ln (3x+2y),\quad x=s\sin t,\quad y=t\cos s$}
\solution 
\begin{align*}
    \pp z s &= \pp z x \bp{\pp x s} + \pp z y\bp{\pp y s} \\
    \pp z s &= \bb{\displaystyle\frac{3}{3x+2y}}(\sin t) +
    \bb{\displaystyle\frac{2}{3x+2y}}(-t\sin s) \\
    \Aboxed{\pp z s &= \displaystyle\frac{3\sin t - 2t\sin s}{3x+2y}}
\end{align*}
\begin{align*}
    \pp z t &= \pp z x \bp{\pp x t} + \pp z y \bp{\pp y t} \\
    \pp z t &= \bb{\displaystyle\frac{3}{3x+2y}}(s\cos t) +
    \bb{\displaystyle\frac{2}{3x+2y}}(\cos s) \\
    \Aboxed{\pp z t &= \displaystyle\frac{3s\cos t + 2\cos s}{3x+2y}}
\end{align*}
\newpage
\subsubsection*{15. $z=(\sin\theta)/r,\quad r=st,\quad\theta = s^2 + t^2$}
\solution
\begin{align*}
    \pp z s &= \pp z r \bp{\pp r s} + \pp z \theta\bp{\pp \theta s} \\
    \pp z s &= \bb{-\displaystyle\frac{\sin\theta}{r^2}}(t) +
    \bb{\displaystyle\frac{\cos\theta}{r}}(2s) \\
    \Aboxed{\pp z s &= -\displaystyle\frac{t\sin\theta}{r^2} +
    \displaystyle\frac{2s\cos\theta}{r}}
\end{align*}
\begin{align*}
    \pp z t &= \pp z r \bp{\pp r t} + \pp z \theta \bp{\pp \theta t} \\
    \pp z t &=
    \bb{-\displaystyle\frac{\sin\theta}{r^2}}(s)+\bb{\displaystyle\frac{\cos\theta}{r}}(2t)
    \\
    \Aboxed{\pp z t &= -\displaystyle\frac{s\sin\theta}{r^2}+
    \displaystyle\frac{2t\cos\theta}{r}}
\end{align*}
\spc
\begin{center}
\subsection*{\textit{25-29 (odd)}}
Use the Chain Rule to find the indicated partial derivatives.
\end{center}
\subsubsection*{
    25.
    $z = x^4 + x^2y,\quad x = s+2t-u,\quad y =stu^2$;\\\\
    $\pp z s, \pp z t, \pp z u\quad \text{when}\;s=4,t=2,u=1$
}
\solution
\begin{align*}
    \intertext{When $s = 4$, $t=2$, and $u=1$ $\rr$ $x=7$, $y=8$}
    \pp z s &= \bb{4x^3 +2xy}(1) + \bb{x^2}(tu^2) = 4x^3 + 2xy + x^2tu^2 \\
    \pp z s &= 4(7)^3 + 2(7)(8) + (7)^2(2)(1)^2 \\
    \Aboxed{\pp z s &= 1582}
\end{align*}
\begin{align*}
    \pp z t &= \bb{4x^3+2xy}(2)+\bb{x^2}(su^2) = 8x^3+4xy+x^2su^2 \\
    \pp z t &= 8(7)^3+4(7)(8)+(7)^2(4)(1)^2 \\
    \Aboxed{\pp z t &= 3164}
\end{align*}
\begin{align*}
    \pp z u &= \bb{4x^3+2xy}(-1)+\bb{x^2}(2stu) = -4x^3-2xy+2x^2stu \\
    \pp z u &= -4(7)^3-2(7)(8)+2(7)^2(4)(2)(1) \\
    \Aboxed{\pp z u &= -700}
\end{align*}
\subsubsection*{27. 
    $w = xy + yz +zx,\quad x = r\cos\theta,\quad y = r\sin\theta,\quad z =
    r\theta$;\\\\
    $\pp w r, \pp w \theta\quad \text{when}\;r=2,\theta = \pi/2$
}
\solution
\begin{align*}
    \intertext{When $r=2$ and $\theta=\pi/2$ $\rr$ $x=0$, $y=2$, $z=\pi$}
    \pp w r &= [y+z](\cos\theta)+[x+z](\sin\theta)+[x+y](\theta) \\
    \pp w r &= [2+\pi](0)+[0+\pi](1)+[0+2](\pi/2) = 0+\pi+\pi \\
    \Aboxed{\pp w r &= 2\pi}
\end{align*}
\begin{align*}
    \pp w \theta &= [y+z](-r\sin\theta)+[x+z](r\cos\theta)+[x+y](r) \\
    \pp w \theta &= [2+\pi](-2\cdot 1)+[0+\pi](0)+[0+2](2) = -4 - 2\pi + 0 + 4
    \\
    \Aboxed{\pp w \theta &= -2\pi}
\end{align*}
\newpage
\subsubsection*{29. 
    $N = \displaystyle\frac{p+q}{p+r},\quad p=u+vw,\quad q=v+uw,\quad
    r=w+uv$\;\\\\
    $\pp N u, \pp N v, \pp N w\quad \text{when}\; u=2,v=3,w=4$
}
\solution
\begin{align*}
    \intertext{When $u=2$, $v=3$, and $w=4$ $\rr$ $p=14$, $q=11$, $r=10$}
    \pp N u &=
    \bb{\displaystyle\frac{(p+r)-(p+q)}{(p+r)^2}}(1)+\bb{\displaystyle\frac{1}{p+r}}(w)+\bb{-\displaystyle\frac{p+q}{(p+r)^2}}(v)
    \\
    \pp N u &=
    \displaystyle\frac{r-q}{(p+r)^2}+\displaystyle\frac{w}{p+r}-\displaystyle\frac{v(p+q)}{(p+r)^2} \\
    \pp N u &=
    \displaystyle\frac{10-11}{(14+10)^2}+\displaystyle\frac{4}{14+10}-\displaystyle\frac{3(14+11)}{(14+10)^2} \\
    \pp N u &= \displaystyle\frac{-1}{576}+\displaystyle\frac{4}{24}-\displaystyle\frac{75}{576} = \displaystyle\frac{-1+96-75}{576} = \displaystyle\frac{20}{576} \\
    \Aboxed{\pp N u &= \displaystyle\frac{5}{144}}
\end{align*}
\begin{align*}
    \pp N v &=
    \bb{\displaystyle\frac{r-q}{(p+r)^2}}(w)+\bb{\displaystyle\frac{1}{p+r}}(1)+\bb{-\displaystyle\frac{p+q}{(p+r)^2}}(u)
    \\
    \pp N v &=
    \displaystyle\frac{w(r-q)}{(p+r)^2}+\displaystyle\frac{1}{p+r}-\displaystyle\frac{u(p+q)}{(p+r)^2} \\
    \pp N v &=
    \displaystyle\frac{4(10-11)}{(14+10)^2}+\displaystyle\frac{1}{14+10}-\displaystyle\frac{2(14+11)}{(14+10)^2} \\
    \pp N v &= \displaystyle\frac{-4}{576}+\displaystyle\frac{1}{24}-\displaystyle\frac{50}{576}
            = \displaystyle\frac{-4+24-50}{576}=\displaystyle\frac{-30}{576} \\
    \Aboxed{\pp N v &= -\displaystyle\frac{5}{96}}
\end{align*}
\begin{align*}
    \pp N w &=
    \bb{\displaystyle\frac{r-q}{(p+r)^2}}(v)+\bb{\displaystyle\frac{1}{p+r}}(u)+\bb{-\displaystyle\frac{p+q}{(p+r)^2}}(1) \\
    \pp N w &=
    \displaystyle\frac{v(r-q)}{(p+r)^2}+\displaystyle\frac{u}{p+r}-\displaystyle\frac{p+q}{(p+r)^2} \\
    \pp N w &=
    \displaystyle\frac{3(10-11)}{(14+10)^2}+\displaystyle\frac{2}{(14+10)}-\displaystyle\frac{14+11}{(14+10)^2}
    \\
    \pp N w &=
    \displaystyle\frac{-3}{576}+\displaystyle\frac{2}{24}-\displaystyle\frac{25}{576} = \displaystyle\frac{-3+48-25}{576}=\displaystyle\frac{20}{576} \\
    \Aboxed{\pp N w &= \displaystyle\frac{5}{144}}
\end{align*}
\newpage
\begin{center}
    Use Equation 5 to find $\dd y x$ 
\end{center}
\subsubsection*{31. $y\cos x = x^2 + y^2$}
\solution 
\begin{align*}
    \dd y x &= -\displaystyle\frac{\pp F x}{\pp F y} = -\displaystyle\frac{F_x}{F_y}\\
    \intertext{Let $F &= y\cos x -x^2 - y^2$}
    F_x &= -y\sin x -2x \\
    F_y &= \cos x -2y \\
    \Aboxed{\dd y x = -\displaystyle\frac{F_x}{F_y} &=
    -\displaystyle\frac{(-y\sin x -2x)}{\cos x -2y} = \displaystyle\frac{2x+y\sin
x}{\cos x -2y}}
\end{align*}
\spc
\begin{center}
    Use Equations 6 to find $\pp z x$ and $\pp z y$ 
\end{center}
\subsubsection*{35. $x^2+2y^2+3z^2=1$}
\solution
\begin{align*}
    \pp z x = -\displaystyle\frac{F_x}{F_z} &\quad\quad \pp z y = -\displaystyle\frac{F_y}{F_z} \\
    \intertext{Let $F = x^2+&2y^2+3z^2-1$}
    F_x = 2x \quad\quad &F_y = 4y \quad\quad F_z = 6z \\
    \Aboxed{\pp z x = -\displaystyle\frac{F_x}{F_z} &=
    -\displaystyle\frac{2x}{6z} = -\displaystyle\frac{x}{3z}} \\
    \Aboxed{\pp z y = -\displaystyle\frac{F_y}{F_z} &=
    -\displaystyle\frac{4y}{6z} = -\displaystyle\frac{2y}{3z}} \\
\end{align*}
\newpage
\begin{center}
    \section*{\underline{Section 6: Directional Derivatives and the Gradient
    Vector}}
\end{center}
\begin{center}
    \subsection*{\textit{5, 7}}
    Find the directional derivative of $f$ at the given point in the direction
    indicated by the angle $\theta$.
\end{center}
\subsubsection*{5. $f(x,y)=xy^3-x^2,\quad(1,2),\quad \theta=\pi / 3$}
\subsubsection*{7. $f(x,y)=arctan(xy),\quad(2,-3),\quad\theta = 3\pi /4$}
\spc 
\begin{align*}
    &\text{(a) Find the gradient of $f$} \\ 
    &\text{(b) Evaluate the gradient at the point $P$} \\
    &\text{(c) Find the rate of change of $f$ at $P$ in the direction of the vector
    $\vec u$}
\end{align*}
\subsubsection*{9. $f(x,y)=x/y,\quad P(2,1),\quad \vec u = \frac{3}{5}\ihat +
\frac 4 5 \jhat$}
\spc 
\begin{center}
    \subsection*{\textit{13, 15}} 
    Find the directional derivative of the function at the given point in the
    direction of the vector $\vec v$.
\end{center}
\subsubsection*{13. $f(x,y)=e^x\sin y,\quad (0,\pi /3),\quad \vec v = \vv{-6, 8}$}
\subsubsection*{15. $g(s,t)= s\sqrt t,\quad(2,4),\quad \vec v = 2\ihat - \jhat$}
\spc
\begin{center}
    \subsection*{\textit{21, 23}} 
    Find the directional derivative of the function at the point $P$ in the
    direction of the point $Q$.
\end{center}
\subsubsection*{21. $f(x,y) = x^2y^2-y^3,\quad P(1,2), \quad Q(-3,5)$}
\subsubsection*{23. $f(x,y) = \sqrt{xy},\quad P(2,8),\quad Q(5,4)$}
\spc 
\begin{center}
    \subsection*{\textit{27, 29}} 
    Find the maximum rate of change of $f$ at the given point and the direction
    in which it occurs.
\end{center}
\subsubsection*{27. $f(x,y) = 5xy^2,\quad (3,-2)$}
\subsubsection*{29. $f(x,y) = \sin (xy),\quad (1,0)$}
\spc 
\begin{center}
    \subsection*{\textit{37}} 
\end{center}
The temperature $T$ in a metal ball is inversely proportional to the
distance from the center of the ball, which we take to be the origin. The
temperature at the point $(1,2,2)$ is $120\degree$
\vspace{1em} \\
\vspace{1em}
(a) Find the rate of change of $T$ at $(1,2,2)$ in the direction
toward the point $(2,1,3)$. \\
(b) Show that at any point in the ball the direction of greatest
increase in temperature is given by a vector that points toward the origin
 \spc
\begin{center}
    \subsection*{\textit{47-51 (odd)}} 
    Find equations of (a) the tangent plane and (b) the normal line to the given
    surface at the specified point.
\end{center}
\subsubsection*{47. $2(x-2)^2 + (y-1)^2 + (z-3)^2 = 10,\quad (3,3,5)$}
\subsubsection*{49. $xy^2z^3=8,\quad (2,2,1)$}
\subsubsection*{51. $x+y+z=e^{xyz},\quad (0,0,1)$}
\newpage
\begin{center}
    \section*{\underline{Section 7: Maximum and Minimum Values}}
\end{center}
\begin{center}
    \subsection*{\textit{5-21 (odd)}} 
\end{center}
Find the local maximum and minimum values and saddle point(s) of the
function. You are encouraged to use a calculator or computer to graph the
function with a domain and viewpoint that reveals all the important aspects
of the function.
\subsubsection*{5. $f(x,y)=x^2 + xy + y^2 + y$}
\subsubsection*{7. $f(x,y)= 2x^2-8xy+y^4-4y^3$}
\subsubsection*{9. $f(x,y)=(x-y)(1-xy)$}
\subsubsection*{11. $f(x,y)=y\sqrt x -y^2-2x+7y$}
\subsubsection*{13. $f(x,y)=x^3-3x+3xy^2$}
\subsubsection*{15. $f(x,y)=x^4 - 2x^2+y^3-3y$}
\subsubsection*{17. $f(x,y)=xy-x^2y-xy^2$}
\subsubsection*{19. $f(x,y)=e^x\cos y$}
\subsubsection*{21. $f(x,y)=y^2-2y\cos x,\quad -1 \leq x \leq 7$}
\spc
\begin{center}
    \subsection*{\textit{33-39 (odd)}} 
    Find the absolute maximum and minimum values of $f$ on the set $D$.
\end{center}
\subsubsection*{33. $f(x,y) = x^2+y^2-2x,\quad$}
$D$ is the closed triangular
region with vertices $(2,0)$, $(0,2)$, and $(0,-2)$
\subsubsection*{35. $f(x,y)=x^2+y^2+x^2y+4,$} 
$D = \{(x,y)\such |x| \leq 1, |y| \leq 1\}$
\subsubsection*{37. $f(x,y)=x^2+2y^2-2x-4y+1,$}
$D = \{(x,y)\such 0\leq x \leq 2,0 \leq y \leq 3\}$
\subsubsection*{39. $f(x,y)= 2x^3+y^4,$}
$D=\{(x,y)\such x^2+y^2 \leq 1\}
\spc 
\begin{center}
    \subsection*{\textit{43}} 
Find the shortest distance from the point $(2,0,-3)$ to the plan $x+y+z=1$. 
\end{center}
\spc
\begin{center}
    \subsection*{\textit{45}} 
Find the points on the cone $z^2 = x^2 + y^2$ that are closest to the point
$(4,2,0)$. 
\end{center}
\spc
\begin{center}
    \subsection*{\textit{47}} 
Find three positive numbers whose sum is 100 and whose product is a maximum.
\end{center}
\spc
\begin{center}
    \subsection*{\textit{55}} 
\end{center}
A cardboard box without a lid is to have a volume of 32,000 $cm^3$. Find the
dimensions that minimize the amount of cardboard used.
\spc
\end{document}
